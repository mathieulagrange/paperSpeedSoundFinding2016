\documentclass[10pt]{article}

\title{Response to Reviewers: \\ AES-D-16-00016
Semantic browsing of sound databases without keywords}

\usepackage{color}
\newcommand{\gl}[1]{\textcolor{red}{Gregoire : #1}}
\newcommand{\nm}[1]{\textcolor{magenta}{Nicolas : #1}}
\newcommand{\ml}[1]{\textcolor{blue}{ Mathieu : #1}}


\begin{document}

\maketitle

We would like to thank the editor and the reviewers for their comments and suggestions. Following these comments, we made several changes to the article, which are summarized here. The next sections list our answers to each of the reviewer’s comments, with references to the revised manuscript where appropriate.

\ml{TODO: put analysis data online}

\section{Responses to Reviewer 1}

\gl{do it}

\begin{enumerate}

\item \emph{[Are] the sounds heard counter was reset for each search? That is, whether the total and unique heard sounds are related to the whole experiment or an average value per target?}

\item \emph{How many target sounds were assessed by each subject?}

\item \emph{Was this controlled by the study or was the study simply time limited?}

\item \emph{Perhaps in the post-hoc analyses it would be useful to know if FSD produces significant results in pair-wise tests with Bonferroni correction (since this is, albeit simple, still a multiple comparison test).}

\item \emph{On page 5, Section 5.2  "...the three metrics are presented on Figure 6. " => should be in Figure 6.}

This typo is now fixed.

\end{enumerate}

\section{Responses to Reviewer 2}


\begin{enumerate}

\item \emph{If the author replicate the same experiment for example using a music collection or a collection of bird songs and the findings were consistent 
this would justify the old title (and make the paper much stronger)}

This is an excellent suggestion that we decided to do by considering an alternative dataset of sounds produced by musical instruments. As shown on Section 6 of the updated paper, the findings are consistent with the ones obtained using the urban sound dataset.

We thus keep the original title.

\item \emph{The references are relatively old. There have been several newer efforts to develop browsing interfaces especially in the music domain. 
For example the Musicream, Musicrainbow and Musicsun interfaces. 
Knees, Peter, et al. "An innovative three-dimensional user interface for exploring music collections enriched." Proceedings of the 14th annual ACM international conference on Multimedia. ACM, 2006.
Lillie, Anita Shen. MusicBox: Navigating the space of your music. Diss. Massachusetts Institute of Technology, 2008.}

Those references are added to the previous work (Section 1).

\item \emph{The uses of Kruskal's normalized stress is interesting but raises the concern that the hierarchical organization improvement might be specific to this approach. It would have been 
really good to have a few other ways to map the audio features to a spatial display (the classic PCA for example or SOM) and then compare the different interfaces.}

We agree that more extensive research should be done regarding the type of features used and the type of projection used. We originally chose the MDS as it has better performance than the PCA and have the advantage of not relying on any parameter setting unlike the SOM approach. 

This latter motivation is added to the manuscript in Section 3. 

Comparing to an optimized SOM display would have been interested, along with the use of other features. That said, we decided to dedicate the complementary experiments to the use of an alternate dataset.

\item \emph{some thing 
as simple as showing the word associated with a category when navigating would probably help. This would be an easy configuration to test and I am surprised the authors [not]
have done so. I suspect that the speed and efficiency would improve.
}

We agree that this configuration would have helped us gaining a more precise understanding of the results. However, we chose to focus our study on completely non verbal displays as stated in the introduction.

Even if this complementary study would have increase the level of knowledge, we decided to focus the complementary experiments to the use of an alternative dataset.

\end{enumerate}

\end{document}
